\documentclass{cv}
\usepackage{comment}

\begin{document}

\newpagecolor{bg}
\pagestyle{empty}

\noindent\textbf{\color{solviolet} \LARGE Alexandre Trendel}\medskip

\noindent\begin{minipage}[t]{0.6\textwidth}
	Ingénieur en développement web et mobile, 4 ans d'expérience. \smallskip
	
	\textbf{À la recherche d'un rôle full-stack à l'étranger à partir de septembre 2021.}

	\smallskip{\color{solviolet}\rule{3cm}{1.5pt}}
\end{minipage}\hfill%
\begin{minipage}[t]{0.3\textwidth}%
	\faicon{map-marker} Nantes, France
	\newline\faicon{phone} %PHONE%
	\newline\faicon{envelope} \href{mailto:%EMAIL%
	}{\link{%EMAIL%
	}}%
	\newline\faicon{github} \href{https://github.com/xou816}{\link{github.com/xou816}} 
	
\end{minipage}

\vspace{1ex}

\section{Compétences}

\noindent%
\begin{minipage}[t]{0.4\textwidth}
	\subsection{Langages, frameworks}

	\begin{center}
		\otag{Java} \otag{Spring} 
		\otag{Javascript} \otag{React} 
		\otag{Node.js} 
		\otag{Swift} \otag{CocoaPods}
		\otag{PHP} \otag{Python} \otag{Rust}
		\otag{Docker} \otag{git} \otag{Unix}
	\end{center}
\end{minipage}\hfill%
\begin{minipage}[t]{0.4\textwidth}
	\subsection{Langues parlées}

	\begin{itemize}
		\item anglais courant (TOEIC 990, 2018)
		\item notions d'allemand
	\end{itemize}
\end{minipage}

\section{Expérience professionnelle}

\experience
{{nov. 2019 -- aujourd'hui}}
{resources/evtech.png}
{e.Voyageurs SNCF}{Ingénieur études et développement}{

Développement de l'application iOS \href{https://apps.apple.com/fr/app/oui-sncf-train-et-bus/id343889987}{\link{\ouisncf{}}}.

\begin{itemize}
	\item conception et études techniques, chiffrage lors de cérémonies agiles dédiées
	\item développement iOS (UIKit) en Swift
	\item développement d'un framework UI déclaratif désormais utilisé en production
	\item design pattern MVVM, et utilisation de RxSwift
	\item qualité : tests unitaires (par DI), automatisation de tests UI via XCTest, suivi de production
	\item intégration continue : compilation, test et export de notre app, via Gitlab CI et Fastlane
\end{itemize}
}

\experience[main]
{{déc. 2016, \faicon{clock-o} 3 ans}}
{resources/evtech.png}
{e.Voyageurs SNCF}{Alternant ingénieur études et développement}{

Développement du nouveau compte client \ouisncf{}. 

\begin{itemize}
	\item conception, puis chiffrage des tâches de développements ; méthodologies agiles (Kanban)
	\item développement de micro-services REST réactifs avec Spring WebFlux
	\item développement d'un batch de synchronisation de bases de données via un bus RabbitMQ
	\item application des principes du Domain Driven Design
	\item développement d'un frontend React/Redux ; contribution à une librairie de composants communs
	\item qualité : tests d'intégration orienté métier (BDD) via Cucumber et Junit
	\item intégration et déploiement continu via Jenkins : tests d'intégration de nos services Dockerisés, 
	déploiement jusqu'à la production (``BlueGreen'')
	\item suivi et analyse de la production (aggrégation de logs et métriques applicatives)
\end{itemize}
}

\experience
{{mai 2018, \faicon{clock-o} 4 mois}}
{resources/adfab.png}
{Adfab (Montréal)}{Project Manager Intern}{

Chef de projet technique au sein d'une petite structure.

\begin{itemize}
	\item rencontre et échange avec les clients (particuliers ou entreprises)
	\item spécification fonctionnelle et technique, rédaction d'user stories
	\item test des user stories
	\item déploiement applicatif vers un environnement AWS
	\item suivi de production
\end{itemize}
}

\pagestyle{withheader}

\section{Formation}

\experience[main]
{{2016 -- 2019}}
{resources/ecn.png}
{Centrale Nantes}{Élève ingénieur généraliste}{

Ingénieur généraliste en alternance.

\begin{itemize}
	\item tronc commun ingénieur
	\item spécialisation maths appliquées/sciences des données (via l'Université de Nantes)
\end{itemize}

Diplôme obtenu en 2019.
}

\section{Projets personnels}

J'aime explorer de nouvelles technos dans mon temps libre. Quelques-uns de mes projets \href{https://github.com/xou816}{\link{Github}} :

\begin{description}[leftmargin=!,labelwidth=3cm]
	\item[spot {\footnotesize(\faStar{}890)}] Client Spotify en Gtk et Rust pour le bureau Linux
	\item[edt-ecn] (inactif) Web app permettant de consulter l'emploi du temps à Centrale Nantes
	\item[silex-autowiring] (inactif) Librairie PHP facilitant l'injection de dépendances pour Silex
\end{description}

\section{Centres d'intérêt}

Cuisinier à domicile, coureur du dimanche, se tremousse sur du post-punk, avide lecteur de science-fiction.

\end{document}