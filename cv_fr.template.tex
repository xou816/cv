\documentclass{cv}
\usepackage{comment}

\begin{document}

\newpagecolor{bg}
\pagestyle{empty}

\noindent\textbf{\color{solviolet} \LARGE Alexandre Trendel}\medskip

\noindent\begin{minipage}[t]{0.5\textwidth}
	Ingénieur en développement web et mobile, 4 ans d'expérience. \smallskip
	
	\textbf{À la recherche d'un rôle full-stack au Canada à partir de septembre 2021.}	
\end{minipage}\hfill%
\begin{minipage}[t]{0.3\textwidth}
	
	\faicon{map-marker} Nantes, France
	\newline\faicon{whatsapp} %PHONE%
	\newline\faicon{envelope} \href{mailto:%EMAIL%
	}{\link{%EMAIL%
	}}%
	\newline\faicon{github} \href{https://github.com/xou816}{\link{github.com/xou816}} 
	
\end{minipage}

\bigskip\noindent{\color{solviolet}\rule{3cm}{1.5pt}}

\section{Compétences}

\noindent%
\begin{minipage}[t]{0.4\textwidth}
	\subsection{Langages, frameworks}
	\begin{center}
		\otag{Java} \otag{Spring} 
		\otag{Javascript} \otag{React} 
		\otag{Node.js} 
		\otag{Swift} \otag{CocoaPods}
		\otag{PHP} \otag{Python} \otag{Rust}
		\otag{Docker} \otag{git} \otag{Unix}
	\end{center}
\end{minipage}\hfill%
\begin{minipage}[t]{0.5\textwidth}
	\subsection{Langues}
	\begin{itemize}
		\item anglais courant
		\item notions d'allemand
	\end{itemize}
	
	\subsection{Compétences humaines}
	\begin{itemize}
		\item curieux, bienveillant
		\item s'adapte très rapidement à de nouveaux contextes
		\item bonnes capacités d'analyse
	\end{itemize}
\end{minipage}

\section{Expérience professionnelle}

\experience[main]
{{juin 2021 -- aujourd'hui}}
{resources/evtech.png}
{e.Voyageurs SNCF}{Ingénieur études et développement}{

Développement full-stack d'une application mobile.

\begin{itemize}
	\item développement cross-platform iOS/Android à l'aide de Flutter
	\item développement d'un backend Kotlin et Spring
	\item intégration continue et déploiement vers le cloud AWS
\end{itemize}
}

\experience
{{nov. 2019, \faicon{clock-o} 1 an et demi}}
{resources/evtech.png}
{e.Voyageurs SNCF}{Ingénieur études et développement}{

Développement de l'application iOS \href{https://apps.apple.com/fr/app/oui-sncf-train-et-bus/id343889987}{\link{\ouisncf{}}}.

\begin{itemize}
	\item conception et études techniques, chiffrage lors de cérémonies agiles dédiées
	\item développement iOS UIKit en Swift et Obj-C
	\item développement d'un framework UI déclaratif désormais utilisé en production
	\item automatisation de tests UI via XCTest
	\item intégration continue via Gitlab CI et Fastlane
	\item design pattern MVVM, utilisation de RxSwift
	\item suivi de production (Firebase, Instana)
\end{itemize}
}

\experience
{{déc. 2016, \faicon{clock-o} 3 ans}}
{resources/evtech.png}
{e.Voyageurs SNCF}{Alternant ingénieur études et développement}{

Développement du nouveau compte client \ouisncf{}. 

\begin{itemize}
	\item conception, puis chiffrage des tâches de développements; utilisation de méthodologies agiles (Kanban)
	\item développement de micro-services REST réactifs avec Spring WebFlux
	\item développement d'un batch de synchronisation de bases de données via un bus RabbitMQ
	\item application des principes du Domain Driven Design
	\item tests d'intégration orienté métier (BDD) via Cucumber/Junit
	\item développement d'un frontend React/Redux; contribution à une librairie de composants communs
	\item intégration et déploiement continu via Jenkins : tests d'intégration de nos services Dockerisés, 
	déploiement jusqu'à la production (``BlueGreen'')
	\item suivi et analyse de la production (aggrégation de logs et métriques applicatives)
\end{itemize}
}

\experience
{{mai 2018, \faicon{clock-o} 4 mois}}
{resources/adfab.png}
{Adfab (Montréal)}{Project Manager Intern}{

Chef de projet technique au sein d'une petite structure.

\begin{itemize}
	\item rencontre et échange avec les clients (particuliers ou entreprises)
	\item spécification fonctionnelle et technique, rédaction d'user stories
	\item test des user stories
	\item déploiement applicatif vers un environnement AWS
	\item suivi de production
\end{itemize}
}

\begin{center}
	\bfseries Références sur demande.
\end{center}

\pagestyle{withheader}

\section{Formation}

\experience[main]
{{2016 -- 2019}}
{resources/ecn.png}
{Centrale Nantes}{Élève ingénieur généraliste}{

Ingénieur généraliste en alternance.

\begin{itemize}
	\item tronc commun ingénieur
	\item spécialisation maths appliquées/sciences des données (via l'Université de Nantes)
\end{itemize}

Diplôme obtenu en 2019.
}

\section{Centres d'intérêt}

Cuisinier à domicile, coureur du dimanche, se tremousse sur du post-punk et avide lecteur de science-fiction.

\end{document}