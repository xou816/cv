\documentclass{cv}
\usepackage{comment}

\begin{document}

\newpagecolor{bg}
\pagestyle{empty}

\noindent\begin{minipage}[t]{0.5\textwidth}

	\textbf{\color{solviolet} \LARGE Alexandre Trendel}\smallskip

	Ingénieur en développement web et mobile, 4 ans d'expérience. \smallskip
	
	\textbf{À la recherche d'un rôle full-stack au Canada à partir de septembre 2021.}

	\color{solviolet}\rule{2cm}{1pt}
	
\end{minipage}\hfill%
\begin{minipage}[t]{0.3\textwidth}

	\colsec{Coordonnées}

	\faicon{map-marker} Nantes, France
	\newline\faicon{phone} %PHONE%
	\newline\faicon{envelope} \href{mailto:%EMAIL%
	}{\link{%EMAIL%
	}}%
	\newline\faicon{github} \href{https://github.com/xou816}{\link{github.com/xou816}} 

\end{minipage}

\colsec{Compétences techniques}

\begin{description}
\item[Dév. backend] Conception et développement de services robustes en Java 8 et 11, framework Spring
\item[Dév. frontend] Développement de SPA/PWA en Javascript/Typescript, via React/Redux
\item[Dév. mobile] Développement natif iOS (UIKit), Swift 5; développement cross-platform (Flutter)
\item[Architecture applicative] Frontend (Flux, MVC...) comme backend (DDD...)
\item[Langues] anglais courant (TOEIC 990), notions d'allemand
\end{description}

\colsec{Compétences humaines}

\textbf{Curieux} et \textbf{bienveillant}, je cherche à apprendre des autres et à m'épanouir techniquement. 
Intéressé par le logiciel libre, je suis également impliqué dans des projets open-source. 
Je sais m'\textbf{adapter très rapidement} à de nouveaux contextes (et j'y prend plaisir !), et j'ai 
de bonnes capacités d'analyse.

\colsec{Expérience}

\experience[main]
{{juin 2021 -- aujourd'hui}}
{resources/evtech.png}
{e.Voyageurs SNCF}{Ingénieur études et développement}{

Développement full-stack d'une nouvelle application mobile.

\begin{itemize}
	\item développement cross-platform \otag{Flutter 2}
	\item développement d'un backend \otag{Kotlin} et \otag{Spring}
	\item intégration continue et déploiement vers le cloud \otag{AWS}
\end{itemize}
}

\experience
{{nov. 2019, \faicon{clock-o} 1 an et demi}}
{resources/evtech.png}
{e.Voyageurs SNCF}{Ingénieur études et développement}{

Développement de l'application iOS \href{https://apps.apple.com/fr/app/oui-sncf-train-et-bus/id343889987}{\link{\ouisncf{}}}.

\begin{itemize}
	\item conception et études techniques
	\item chiffrage lors de cérémonies agiles dédiées
	\item développement iOS (UIKit) en \otag{Swift 5} et Obj-C
	\item consommation de services JSON
	\item développement d'un framework UI déclaratif désormais utilisé en production
	\item automatisation de tests UI via XCTest
	\item intégration continue: compilation, test et export de notre app, via \otag{Gitlab CI} et Fastlane
	\item design pattern MVVM, et utilisation de RxSwift
	\item suivi de production (Firebase, Instana)
\end{itemize}
}

\experience
{{déc. 2016, \faicon{clock-o} 3 ans}}
{resources/evtech.png}
{e.Voyageurs SNCF}{Alternant ingénieur études et développement}{

Développement du nouveau compte client \ouisncf{}.	

\begin{itemize}
	\item conception, puis chiffrage des tâches de développements; utilisation de méthodologies agiles (Kanban)
	\item développement de micro-services REST réactifs avec \otag{Spring WebFlux}
	\item développement d'un batch de synchronisation de bases de données via un bus \otag{RabbitMQ}
	\item application des principes du Domain Driven Design
	\item tests d'intégration orienté métier (BDD) via \otag{Cucumber/Junit}
	\item développement d'un frontend \otag{React/Redux}; contribution à une librairie de composants communs
	\item intégration et déploiement continu via Jenkins : tests d'intégration de nos services Dockerisés, 
	déploiement jusqu'à la production (``BlueGreen'')
	\item suivi et analyse de la production (aggrégation de logs et métriques applicatives)
\end{itemize}
}

\experience
{{mai 2018, \faicon{clock-o} 4 mois}}
{resources/adfab.png}
{Adfab (Montréal)}{Project Manager Intern}{

Chef de projet technique au sein d'une petite structure.

\begin{itemize}
	\item rencontre et échange avec les clients (particuliers ou entreprises)
	\item spécification fonctionnelle et technique, rédaction d'user stories
	\item test des user stories
	\item déploiement applicatif vers un environnement AWS
	\item suivi de production
\end{itemize}
}

\colsec{Formation}

\experience[main]
{{2016 -- 2019}}
{resources/ecn.png}
{Centrale Nantes}{Élève ingénieur généraliste}{

Ingénieur généraliste en alternance.

\begin{itemize}
	\item tronc commun ingénieur
	\item spécialisation maths appliquées/sciences des données (via l'Université de Nantes)
\end{itemize}
}

\colsec{Centres d'intérêt}

Cuisinier à domicile, coureur du dimanche, se tremousse sur du post-punk et avide lecteur de science-fiction.

\vspace{1cm}
\begin{center}
	\bfseries Références sur demande.
\end{center}

\end{document}