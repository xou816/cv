\documentclass{cv}
\usepackage{comment}

\begin{document}

\newpagecolor{bg}
\pagestyle{empty}

\noindent\begin{minipage}[t]{0.5\textwidth}
	\noindent\textbf{\color{solviolet} \LARGE Alexandre Trendel}\medskip

	Développeur web et mobile, 4 ans d'expérience. \smallskip
	
	\textbf{À la recherche d'un rôle full-stack à Montréal à partir de septembre 2021.}	

	\smallskip\noindent{\color{solviolet}\rule{3cm}{1.5pt}}

\end{minipage}\hfill%
\begin{minipage}[t]{0.3\textwidth}
	\faicon{map-marker} Nantes, France
	\newline\faicon{whatsapp} %PHONE%
	\newline\faicon{envelope} \href{mailto:%EMAIL%
	}{\link{%EMAIL%
	}}%
	\newline\faicon{github} \href{https://github.com/xou816}{\link{github.com/xou816}} 
	
\end{minipage}


\section{Compétences}

Curieux et bienveillant, je m'adapte très rapidement à de nouveaux contextes. J'aime découvrir de nouvelles technologies et maintient des projets en open-source sur \href{https://github.com/xou816}{\link{Github}}.

\noindent%
\begin{minipage}[t]{0.40\textwidth}

	\subsection{Langages et outils}
	\begin{center}
		\otag{Java} \otag{Kotlin} \otag{Spring} 
		\otag{Javascript} \otag{React} 
		\otag{Node.js} 
		\otag{Swift} \otag{CocoaPods}
		\otag{Rust}
		\otag{Docker} \otag{git} \otag{Unix}
	\end{center}

\end{minipage}\hfill%
\begin{minipage}[t]{0.45\textwidth}
	\subsection{Langues}
	Français maternel
	
	Anglais courant {\itshape (TOIEC 990, 2018)}
	
	Notions d'allemand
\end{minipage}

\section{Expérience professionnelle}

\experience[main]
{{juin 2021, \faicon{clock-o} 3 mois}}
{resources/evtech.png}
{e.Voyageurs SNCF}{Ingénieur études et développement}{Nantes, France}{

Développement \textbf{full-stack} d'une nouvelle application web et mobile.

\begin{itemize}
	\item développement iOS/Android en Flutter et Dart
	\item développement d'une web-app Next.js
	\item développement d'un BFF (backend-for-front) Kotlin et Spring WebFlux, gestion de dépendances à l'aide de Gradle
	\item tests unitaires et d'intégration: golden tests (Flutter), Cypress (JS), JUnit (backend)...
	\item intégration continue et déploiement vers AWS via Terraform
\end{itemize}
}

\experience
{{nov. 2019, \faicon{clock-o} 1 an et demi}}
{resources/evtech.png}
{e.Voyageurs SNCF (anciennement Oui.SNCF)}{Ingénieur études et développement}{Nantes, France}{

Développement de l'application iOS \href{https://apps.apple.com/fr/app/oui-sncf-train-et-bus/id343889987}{\link{\ouisncf{}}}.

\begin{itemize}
	\item conception et études techniques, chiffrage lors de cérémonies agiles dédiées
	\item développement iOS UIKit en Swift et Obj-C
	\item développement d'un framework UI déclaratif désormais utilisé en production
	\item automatisation de tests UI via XCTest
	\item intégration continue via Gitlab CI et Fastlane
	\item design pattern MVVM, utilisation de RxSwift
	\item suivi de production (Firebase, Instana)
\end{itemize}
}

\experience
{{déc. 2016, \faicon{clock-o} 3 ans}}
{resources/oui.jpg}
{Oui.SNCF}{Alternant ingénieur études et développement}{Nantes, France}{

Développement du nouveau compte client \ouisncf{}. 

\begin{itemize}
	\item conception, chiffrage des tâches de développements; méthodologies agiles (Kanban)
	\item développement de micro-services REST réactifs avec Spring WebFlux
	\item développement d'un batch de synchronisation de bases de données via un bus RabbitMQ
	\item application des principes du Domain Driven Design
	\item tests d'intégration orienté métier (BDD) via Cucumber/Junit
	\item développement d'un frontend React/Redux; contribution à une librairie de composants communs
	\item intégration et déploiement continu via Jenkins : tests d'intégration de nos services Dockerisés, 
	déploiement jusqu'à la production (``BlueGreen'')
	\item suivi et analyse de la production (aggrégation de logs et métriques applicatives)
\end{itemize}
}

\experience
{{mai 2018, \faicon{clock-o} 4 mois}}
{resources/adfab.png}
{Adfab}{Project Manager Intern}{Montréal, Québec}{

Chef de projet technique au sein d'une petite structure, en contact avec des entreprises et particuliers de la région de Montréal (Axa Assurances, Devimco, Bota Bota...).

\begin{itemize}
	\item rencontre et échange avec les clients
	\item spécification fonctionnelle et technique, rédaction d'user stories
	\item test des user stories
	\item déploiement vers un environnement AWS
	\item suivi de production
\end{itemize}
}

\begin{center}
	\bfseries Références sur demande.
\end{center}

\pagestyle{withheader}

\section{Formation}

\experience
{{2016 -- 2019}}
{resources/ecn.png}
{Centrale Nantes}{Élève ingénieur généraliste}{Nantes, France}{

Ingénieur généraliste en alternance, équivalent Maîtrise.

\begin{itemize}
	\item tronc commun ingénieur
	\item spécialisation maths appliquées/sciences des données (via l'Université de Nantes)
\end{itemize}

Diplôme obtenu en 2019.
}

\section{Centres d'intérêt}

Cuisinier à domicile, coureur du dimanche, se tremousse sur du post-punk et avide lecteur de science-fiction.

\end{document}